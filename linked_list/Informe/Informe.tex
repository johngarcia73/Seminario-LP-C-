\documentclass{article}
\usepackage[utf8]{inputenc}  % Para manejo de acentos y caracteres UTF-8
\usepackage[T1]{fontenc}     % Para una mejor codificaci\'on de fuentes
\usepackage{lmodern}         % Fuente mejorada
\usepackage{hyperref}        % Para hiperv\'inculos

\title{Conceptos Avanzados en C++}
\author{John García Muñoz\\ Victor Hugo Pacheco Fonseca\\ Jose Agustín del Toro González}
\date{\today}

\begin{document}

\maketitle

\section{Introducción}
En este documento se revisan diversos aspectos fundamentales de C++ relacionados con la gestión de recursos y la inferencia de tipos, enfatizando las mejoras introducidas a partir de C++11. Se abordan temas como el \emph{ownership}, el principio RAII, la inicialización uniforme con \texttt{\{\}}, el uso de expresiones lambda, la diferenciación entre punteros inteligentes y punteros crudos, la palabra clave \texttt{noexcept} y finalmente, los mecanismos de inferencia de tipo mediante \texttt{auto}, \texttt{decltype} y \texttt{decltype(auto)}.

\section{Ownership y Manejo de Recursos}
El concepto de \textbf{ownership} en C++ se refiere a la responsabilidad de administrar la vida útil y la liberación de recursos, tales como la memoria dinámica, archivos abiertos, conexiones de red, entre otros. La correcta gestión del ownership es esencial para evitar errores comunes como las fugas de memoria o el acceso a recursos no válidos.

Para abordar estos desafíos, C++ moderno dispone de herramientas avanzadas que incluyen punteros inteligentes y el uso del principio RAII, el cual permite que la adquisición y liberación de los recursos esté directamente ligada a la existencia de objetos.

\section{RAII y Fugas de Memoria}
\subsection{RAII (Resource Acquisition Is Initialization)}
El principio RAII consiste en asociar la adquisición de un recurso a la inicialización de un objeto. De esta forma, al destruirse el objeto (al salir de su ámbito), se ejecuta su destructor, el cual libera el recurso adquirido. Este mecanismo automatiza la gestión de recursos y disminuye significativamente el riesgo de errores al evitar la liberación manual, lo que puede provocar problemas en el manejo de la memoria y otros recursos críticos.

\subsection{Fugas de Memoria}
Una \textbf{fuga de memoria} ocurre cuando la memoria asignada dinámicamente no se libera adecuadamente. Este error puede provocar que el programa consuma progresivamente más memoria, llevando, en última instancia, a fallos de ejecución o a una disminución en el rendimiento del sistema. La utilización de RAII y de punteros inteligentes es fundamental para prevenir este tipo de problemas, ya que garantiza la liberación oportuna de los recursos.

\section{Inicialización Uniforme: Uso de \texttt{\{\}} versus \texttt{( )}}
Con la introducción de C++11 se implementó la inicialización uniforme mediante el uso de llaves (\texttt{\{\}}). Esta forma de inicialización ofrece una sintaxis coherente y segura que se asemeja al estilo de inicialización de colecciones en lenguajes como C\#. Entre sus ventajas se destaca la prevención de ambigüedades y de conversiones implícitas peligrosas (por ejemplo, las \emph{narrowing conversions}). Por otro lado, la inicialización tradicional mediante paréntesis (\texttt{( )}) continúa siendo válida, pero puede dar lugar a situaciones ambiguas, como la dificultad para diferenciar entre la declaración de una función y la creación de una instancia de objeto.

\section{Expresiones Lambda y su Uso con Algoritmos}
Las expresiones lambda, introducidas en C++11, permiten definir funciones anónimas en línea, lo que resulta especialmente útil para utilizarse junto a algoritmos de la STL, tales como \texttt{std::for\_each}. Estas expresiones posibilitan la incorporación directa de lógica en la iteración de contenedores, permitiendo capturar variables locales tanto por valor como por referencia y, de este modo, lograr soluciones concisas y expresivas.

\section{Punteros Crudos (Raw Pointers)}
A pesar de que la práctica recomendada es el uso de punteros inteligentes (\texttt{std::unique\_ptr}, \texttt{std::shared\_ptr} y \texttt{std::weak\_ptr}) para gestionar el ownership, existen ciertos escenarios en los cuales el uso de punteros crudos es adecuado:
\begin{itemize}
    \item Cuando se necesita una referencia temporal a un objeto sin asumir la responsabilidad de liberar el recurso.
    \item En la interoperabilidad con APIs o bibliotecas heredadas que requieren punteros tradicionales.
    \item En contextos críticos en los que se desea evitar el sobrecoste asociado a la gestión automática y se tiene control explícito sobre la validez del objeto referenciado.
\end{itemize}
En estos casos, el puntero crudo se utiliza únicamente como herramienta de acceso, dejando la gestión de la vida del objeto a otros mecanismos.

\section{\texttt{noexcept}}
La palabra clave \textbf{noexcept} se introdujo en C++11 para declarar que determinadas funciones no lanzarán excepciones. Esta especificación no solo mejora la claridad semántica del código, sino que también permite al compilador realizar optimizaciones, sobre todo en operaciones de movimiento. Es importante utilizar \texttt{noexcept} solo cuando se tiene la certeza de que la función no emitirá ninguna excepción, ya que en caso contrario se invoca \texttt{std::terminate()}, lo que provoca la finalización abrupta del programa.

\section{Inferencia de Tipos en C++}
La inferencia de tipos es una característica que permite al compilador deducir automáticamente el tipo de una variable o el tipo de retorno de una función, lo cual simplifica la escritura del código y lo hace más legible.

\subsection{\texttt{auto}}
La palabra clave \textbf{auto} permite que el compilador deduzca el tipo de una variable basándose en el valor con el que se inicializa. Esta característica es especialmente valiosa cuando se usan tipos complejos o cuando se desea reducir la verbosidad del código. Cabe destacar que, durante la deducción, se omiten las referencias y ciertos calificadores a menos que se indiquen explícitamente.

\subsection{\texttt{decltype}}
Con \textbf{decltype} se puede obtener el tipo exacto de una expresión, preservando todas sus características, incluidos los calificadores \texttt{const} y las referencias. Este mecanismo es muy útil para deducir tipos de retorno en funciones o para trabajar con expresiones cuyo tipo exacto debe mantenerse sin modificaciones.

\subsection{\texttt{decltype(auto)}}
Introducido en C++14, \textbf{decltype(auto)} combina los comportamientos de \texttt{auto} y \texttt{decltype} en la deducción del tipo. Esta sintaxis permite que se mantenga la precisión del tipo deducido, incluyendo las referencias y la categoría de valor (lvalue o rvalue), lo cual resulta especialmente útil al definir funciones con retorno automático.

\section{Alias para Simplificar Nombres de Tipos}
El uso de alias es una práctica recomendada en C++ para simplificar nombres complejos, en particular cuando se trabaja con punteros inteligentes o tipos generados por plantillas. Al definir un alias (o alias template), se otorga un nombre más corto y significativo a un tipo, lo que facilita la legibilidad y el mantenimiento del código sin alterar la semántica subyacente.

\section{Conclusión}
C++ ha evolucionado de manera significativa con la incorporación de nuevas herramientas y paradigmas que permiten una gestión de la memoria más segura, eficiente y expresiva. La implementación de punteros inteligentes, el uso del principio RAII, la inicialización uniforme mediante llaves, la incorporación de expresiones lambda, la especificación de \texttt{noexcept} y los mecanismos de inferencia de tipos son elementos esenciales para escribir código moderno y robusto. Estos conceptos, que favorecen tanto la seguridad como el rendimiento, consolidan a C++ como un lenguaje de alto control y eficiencia, a la par de otros lenguajes modernos como C\# en lo que se refiere a la claridad y concisión en la inicialización de objetos.

\end{document}